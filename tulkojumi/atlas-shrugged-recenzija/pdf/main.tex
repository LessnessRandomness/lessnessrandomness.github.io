\documentclass[12pt]{article}
\usepackage[utf8]{inputenc}
\usepackage[margin=2cm]{geometry}
\usepackage{graphicx}
\graphicspath{ {./images/} }
\usepackage{indentfirst}
\frenchspacing
\usepackage{dirtytalk}
%\usepackage{showframe}
\usepackage{hyperref}

\title{\say{Atlas Shrugged} - recenzija}
\author{Adam Lee}
\date{2013 - 2016}

\begin{document}

\maketitle

\tableofcontents

\section{Priekšvārds: Novele vienam procentam}
\label{sec:foreword}

Tālajā 2011. gadā izlasīju Ainas Rendas magnum opus \say{Atlas Shrugged} - visas 1074 lapaspuses~- ar solījumu, ka galu galā pieķeršos un uzrakstīšu recenziju nodaļu pēc nodaļas, līdzīgi recenzijai par Lī Strobela \href{https://web.archive.org/web/20160313005416/http://www.patheos.com/blogs/daylightatheism/series/the-case-for-a-creator}{\say{The Case for a Creator}}. Mani ilgu laiku aizkavēja citi projekti, bet nu pienācis laiks izpildīt šo solījumu.

Ja tu neesi izlasījis \say{Atlas Shrugged} no vāka līdz vākam (un kurš var tevi vainot?), lūk īss sižets. Darbība notiek \href{https://tvtropes.org/pmwiki/pmwiki.php/Main/TwentyMinutesIntoTheFuture}{divdesmit minūtes nākotnē} ASV, pasaule kļūst aizvien vairāk komunistiska, kā rezultātā sabrūk ekonomika. Neliela saujiņa cilvēku, kas vēl tic kapitālismam, lielākoties industrijas kapteiņi, pasludina, ka viņi vairs nevēlas uzturēt nepateicīgās masas ar savu darbu un piesaka streiku, atņemot savus produktīvos talantus pasaulei, kas nevēlas viņiem atbilstoši atlīdzināt.

Tu varbūt nodomāsi, ka grāmata varētu beigties, visiem citiem saprotot savu kļūdu un lūdzoties kapitālistus atgriezties un viņus izglābt, taču nē. Beigās civilizācija sabrūk, miljoniem cilvēku nomirst bada nāvē un daži laimīgie dzīvo komfortablā nošķirtā patvērumā kalnos, izolēti no visur esošā haosa un anarhijas. Pēc Rendas domām šīs ir laimīgas beigas.

\say{Atlas Shrugged} nekaunas būt novele, kas ir par vienu procentu un ir domāta tam, un, patīk tas vai nē, tāpēc tas ir darbs, kas domāts mūsu laikam un ir par to. Tā argumentē, ka mežonīgais kapitālisms, ko neierobežo noteikumi vai regulējumi, ir ne tikai labākais, bet arī \textit{vienīgais} sabiedrības organizācijas veids, un nodokļi, likumi un sociālās programmas ir nepiedodami ierobežojumi dažu indivīdu nešaubīgajām tiesībām kļūt pēc iespējas bagātākiem. Rendas pasaules uzskats ir maniheistisks, pilnībā melnbalts: vai nu tu esi varonīgi savtīgs kapitālists (viņa uzskatīja savtīgumu par augstāko no visiem tikumiem), vai arī tu esi viens no izlaupītājiem. Un viņa nekautrējas apgalvot - cilvēki, kas netic kapitālismam, kā viņa to definē, ir ne tikai slinki, bet arī \textit{nevērtīgi}: diedelnieki, parazīti, burtiski tik ļoti dzīves necienīgi, ka par viņu nāvēm ir jāgavilē.

Par spīti šim asinis stindzinoši cietsirdīgajam morālam uzskatam, Rendas retorika par da\-rī\-tā\-jiem un ņēmējiem ir vairāk nekā jebkad populāra pašidentificētu konvervatīvo vidū. Pēdējo divu prezidenta vēlēšanu kampaņu laikā, Tējas Partijas pārstāvji dusmīgi  \href{https://web.archive.org/web/20160309004804/http://opinionator.blogs.nytimes.com/2009/03/06/going-galt-everyones-doing-it/?ref=opinion}{draudēja rīkoties kā Galts un liegt mums visiem viņu produktīvo diženumu}. Pauls Raiens izteicās, ka viņš \href{https://web.archive.org/web/20160421063828/http://www.thewire.com/politics/2012/04/how-tell-paul-ryan-wants-be-veep-hes-rejected-his-former-idol-ayn-rand/51605}{kādreiz pieprasīja saviem kongresa darbiniekiem lasīt Rendu}, un Mita Romnija komentārs par 47\% no populācijas, kas ir slinki liekēži, varētu būt nācis pa tiešo no \say{Atlas Shrugged}, lai gan grāmatā šāds tēls būtu bijis varonis, nevis ļaundaris. Un visbeidzot, lai gan ne mazāk svarīgi,  \href{https://web.archive.org/web/20160309004804/http://www.rightwingwatch.org/content/independence-park-be-glenn-becks-galts-gulch}{Glens Beks paziņoja par savu grandiozo plānu būvēt Rendas cienīgu kapitālistu utopiju}, reālās dzīves Galta Aizu, kurā valdītu brīvība - tavā rīcībā tikai par 2 miljardiem dolāru.

Bet nu to droši vien varēja sagaidīt. Tevi vienmēr cildinās, ja teiksi bagātajiem un varenajiem to, ko viņi vēlās dzirdēt, un Renda saka, ka viņu lielā bagātība ir pierādījums viņu \textit{pārākumam} pār citiem. Bagātība un sasniegumi ir taustāmi pierādījumi par morālu pārākumu. Patiesībā viņa iet pat tālāk: viņa argumentē, ka bagātie ir cildeni supercilvēki, līdzīgi senatnes karaļiem-dieviem, un ka visi citi niecīgie cilvēki ir bezcerīgi atkarīgi no viņiem. Laikā, kad pieaug nevienlīdzība, korporāciju glābšana un apjomīgi ielu protesti, kad bagātie var justies kaut nedaudz aplenkti un spiesti aizstāvēties, droši vien nav nekāds pārsteigums, ka viņi vēl ciešāk turas pie Rendas piedāvātās attaisnojošās retorikas, lai atvairītu kritiku.

Tā kā vēlos šo recenziju kādreiz pabeigt, tā būs kā ātrs ceļojums. Nekomentēšu katru grāmatas lapaspusi - apkopošu, kur iespējams, un visticamāk izlaidīšu garlaicīgos monologus - taču pa ceļam būs daudz interesantu pieturas punktu. Ja tev ir pa rokai šī grāmata, vari droši sekot līdzi. Un tagad, ja esi gatavs, aiziet…

\end{document}
